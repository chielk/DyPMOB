% vim: set spell : spelllang=en_gb
\author{Chiel Kooijman - 5743028}
\title{Optimising Communication over a Multiple Access Broadcasting Channel
with a Multi-Dimensional Reward}
\documentclass{article}
\usepackage{graphicx,amsmath}
\begin{document}
	\maketitle
	\section{Task Description}
	\label{sec:task_description}
		The task at hand is one of communication over a single channel.
		Multiple agents broadcast messages, but only one message can be sent at
		any time, otherwise conflicts will arise and no message will come
		through. After a message is sent, the agents will know whether a message
		was sent, a conflict has arisen, or no message was sent.
		Agents have a single buffer that may contain a message. At any concrete
		time-step an agent $i$ with an empty buffer may get a new message with
		probability $p_i(m)$.
		One of the goals is to optimise the total throughput of all the agents.
		This problem has been solved for POMDPs \cite{ooi1996decentralized}, but in this case we would
		also like to avoid dominance of a single agent over the channel.
		By representing the reward as a vector, we can calculate the optimal
		policy $\pi^*$, for any weight vector $\vec{w}$ if we know the reward
		vector $\vec{r}$. The scalar reward $r = \vec{w} \cdot \vec{r}$

		There are several ways to represent throughput and dominance in a
		vector. One would be to have a two-dimensional vector that contains the
		total throughput and some measure of dominance, for instance entropy or
		variance. An advantage of this approach is that the size of the reward
		vector is independent of the number of agents.

		Another way is to represent the vector as an $n$-dimensional vector,
		where $n$ is the number of agents, and each value represents the
		throughput of the corresponding agent. An advantage is that the vector
		contains more information, but there are many more states when the number
		of agents increases. It does however allow for prioritising messages (by
		changing the one of the weights), and the number of states can be greatly
		reduced, because every vector is equal to all of its permutations
		(provided we reorder the state vector and the reward vector in the same
		way).

		In a communication setting, a collision would have the same effect as
		when no agent would send a message. In some settings (such as traffic),
		however a collision may be less desirable. In those cases an extra
		dimension should be added for the number of collisions.
	% section task_description (end)
		\bibliographystyle{apalike}
		\bibliography{references}
\end{document}
